\documentclass[a4paper,12pt]{article}
\usepackage{pdflscape}
\usepackage[utf8]{inputenc}
\usepackage{amsmath}
\usepackage{graphicx} % \scalebox
\usepackage{environ}
\pagenumbering{gobble} 
\title{Bayesian Theorem}
\NewEnviron{myequation}{%
\begin{equation}
\scalebox{5.5}{$\BODY$}
\end{equation}
}
\begin{document}
\begin{landscape}
%% Don't leave this line empty
\begin{myequation}%
P(A_i|\textbf{B}) = \frac{P(A_i) \cdot P(\textbf{B}|A_i)} {P(\textbf{B})}%
\end{myequation}
%
\newline
$P(A_i|\textbf{B})$ désigne la probabilité à postériori, c'est à dire le degré de confiance après la prise en compte des observations
\newline
$P(A_i)$ désigne la probabilité à priori, c'est à dire le degré de confiance que l'on a vis-à-vis de l'hypothèse $A$
\newline
$P(\textbf{B}|A_i)$ désigne vraisemblance, c'est à dire le degré de compatibilité de l'hypothèse $A$ et des observations $\textbf{B}$
\newline
$P(\textbf{B})$ désigne la probabilité d'observation, c'est à dire la probabilité que l'observation survienne dans un cas de figure quelconque
\end{landscape}
\end{document}
